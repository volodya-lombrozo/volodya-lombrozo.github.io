\documentclass{vl}

\begin{document}

    \vlPrintPhoto{}

    \section*{\Large \nospell{Vlad\'imir Zakh\'arov}}

    \href{mailto:volodya.lombrozo@gmail.com}{volodya.lombrozo@gmail.com}\\%
    \href{https://github.com/volodya-lombrozo}{GitHub} /
    \href{https://stackoverflow.com/users/10423604/volodya-lombrozo}{StackOverflow} /
    \href{https://www.linkedin.com/in/vladimir-zakharov-lombrozo-b71744216/}{LinkedIn}

    \vspace*{24pt}

    \href{https://catalog-education.oracle.com/pls/certview/sharebadge?id=87F6A2FE819A5A5AF4120A05900AB28A461EE9A3EE9FBFA02721FADAEB3BCE19}{Oracle-certified}
    developer and architect with over ten years of experience in engineering complex systems
    and deep expertise in object-oriented programming.
    Key developer in several successful projects, including those for Huawei.
    \href{https://github.com/volodya-lombrozo?tab=repositories}{Creator} and
    \href{https://github.com/volodya-lombrozo}{active contributor} to open-source initiatives, such as
    \href{https://github.com/volodya-lombrozo/jtcop}{jtcop},
    \href{https://github.com/volodya-lombrozo/jsmith}{jsmith},
    \href{https://github.com/volodya-lombrozo/xnav}{xnav}.
    One of the core developers of \href{https://www.eolang.org}{EO}, an object-oriented programming language.
    Regular contributor to the StackOverflow community
    (\href{https://stackoverflow.com/users/10423604/volodya-lombrozo}{profile: 3K}) and
    author of patent applications, scientific \href{https://arxiv.org/abs/2410.05631}{publications}, and technical
    \href{https://dzone.com/users/4993224/volodya-lombrozo.html}{articles}.

    \subsection*{Recent Employment}

    \textbf{\href{https://www.huawei.com}{Huawei Technologies Co., Ltd.}} (Moscow, Russia), 12/2021--present.
    Researcher and Developer in the \textbf{Software Development Tools Cloud Technology Lab} and the
    \textbf{Database Intelligence and Optimization Technology Center}

    \begin{itemize}
        \itemsep0em
        \item Developed key components of the \href{https://github.com/objectionary/eo}{EO} programming language;
        \item Developed a disassembler and decompiler from Java bytecode to EO language;
        \item Designed and implemented an enterprise-wide profiling system for Huawei Cloud;
        \item Integrated profiling tools into a large-scale enterprise infrastructure, enhancing system
        observability and diagnostics;
        \item Conducted technical seminars and workshops to educate colleagues on modern software development techniques;
        Reviewed and presented research articles on programming language design and implementation;
        \item Authored scientific articles and patents on programming language design and implementation,
        contributing to academic and industrial research.
    \end{itemize}

    \textbf{\href{https://altinntech.com/en/}{LLC Alternative Innovative Technologies}} (Saint Petersburg, Russia),
    07/2015--12/2021.
    Lead Software Developer.

    \begin{itemize}
        \itemsep0em
        \item Designed and developed enterprise-wide complex information systems, ensuring scalability and reliability;
        \item Developed high-performance Java applications.
        \item Led system architecture decisions, overseeing requirements analysis, design, implementation, testing,
        and maintenance;
        \item Configured monitoring, logging, and CI/CD pipelines, automating deployment and improving system stability;
        \item Advocated and implemented best practices for troubleshooting, monitoring, testing, and deploying
        Java applications;
    \end{itemize}

    \subsection*{Recent Projects}

    \textbf{jeo-maven-plugin (Russia, 2023--present)}.
    A Maven plugin designed to disassemble Java bytecode into the EO programming language and reassemble EO code back
    into Java bytecode.
    Authored and architected the project, leading the development of its core functionalities.
    Implemented features to convert compiled bytecode files into EO language and revert EO code back to bytecode.
    Enhanced the plugin's performance, conducted comprehensive code reviews and improved testing coverage to maintain
    code quality.
    Facilitated technical seminars and workshops to educate team members on the plugin's usage and underlying
    technologies.
    Java, EO, Maven, ASM, jcabi-xml, Saxon-HE, GitHub.

    \textbf{opeo-maven-plugin (Russia, 2023--2024)}.
    A Maven plugin developed to decompile Java programs by transforming low-level EO instructions
    generated by the jeo-maven-plugin into high-level EO language constructs and vice versa.
    Authored and architected the project, leading the development of its core functionalities.
    Implemented features, ensured seamless integration with the jeo-maven-plugin, adapting to API changes and
    maintaining compatibility.
    Conducted comprehensive code reviews and improved testing coverage to maintain code quality.
    Facilitated technical seminars and workshops to educate team members on the plugin's usage and underlying
    technologies.
    Java, EO, Maven, jcabi-xml, GitHub.

    \textbf{EO language (Russia, 2022--present)}.
    A new object-oriented programming language
    that rethinks the very idea of objects and suggests a set of new OOP principles.
    Participated in the development of several core language components,
    including the compiler, runtime, and package manager.
    Improved code quality and testing coverage within the project.
    Fixed many performance issues and hard-to-detect concurrency bugs.
    Developed a translator and decompiler from bytecode to EO language.
    Wrote a patent application related to new methods of managing dependencies in programming languages.
    Wrote an article dedicated to the usage of OOP within Java projects.
    Participated in code reviews.
    Provided technical seminars and workshops within the team.
    Java, JS, EO, XSLT, XSD, XPath, ANTLR4, TeX, GitHub, Open Source.

    \textbf{Java Profiling System (Huawei, 2021--2022)}.
    Enterprise-wide profiling system for Huawei Cloud.
    Developed and integrated profiling tools within large enterprise infrastructure.
    Researched modern profiling techniques and tools and integrated them into the system.
    Worked with the distributed international team of developers.
    Provided code reviews.
    Java, Go, Mongodb, Docker, GitLab, Linux.

    \textbf{\href{http://zillion.games/}{Zillion Games} (Russia, 2020)}.
    Game platform (games, management and configuration systems, analytical systems, and payment processing systems.)
    Developed most of the components and microservices.
    Troubleshot and fixed complex problems, including performance issues and hardly detectable concurrency bugs.
    Led a team of software developers.
    Integrated code quality and testing practices in the team.
    Java, Microservices, Spring, Spring-AMQP, Hibernate, JDBC, Netty, RabbitMQ, JUnit, Mockito, Graylog, Zabbix, Docker,
    Bitbucket, Jenkins, SonarQube, Jira, Confluence.

    \textbf{E.Clean car wash (Russia, Kazakhstan, 2019)}.
    An Android application that allows users to order a car wash based on their location.
    Developed and deployed the mobile app on the Play Store.
    Developed and deployed the backend part of the system on dedicated Linux servers.
    Led a team of software developers.
    Java, Jakarta EE, JSF, JPA, JAX-RS, Hibernate, MySQL, PrimeFaces, JUnit, Mockito, Android, Linux.

    \textbf{Order Management System for "Staut" company (Russia, 2018)}.
    Developed a system to manage the procurement of components and equipment.
    Designed, implemented, deployed, and maintained both backend and frontend components.
    Built a robust and maintainable test suite, participated in code reviews, and led a team of several software developers.
    Java, Java‐EE, JSF, JPA, Hibernate, MySQL, Apache-Poi, JUnit, Mockito, Jenkins, Linux.

    \textbf{Analytical System for "Cytomed" Company (Russia, 2018)}.
    Analytical System for \href{https://cytomed.ru/en/}{Cytomed} company that automates the process of tracking sales
    made by the company.
    Designed, developed, deployed, and supported the entire system.
    Built a maintainable and efficient layers of tests.
    Participated in code reviews.
    Lead a team of several software developers.
    Java, Java‐EE, Vaadin, JPA, Hibernate, MySQL, Jasper Reports, Apache-Poi, JUnit, Mockito, Jenkins, Linux.

    \textbf{Terminal Network Management System (Russia, 2017)}.
    Complex payment system that works with terminals that can accept payments, credit money, and debit funds,
    as well as issue cash banknotes from user personal accounts.
    Designed and developed a web application for the management of remote terminal systems.
    Built a maintainable and efficient layer of tests that ensured stable and reliable operation of the system.
    Java, Java EE 7, JSF, PrimeFaces, JPA, Hibernate, MySQL, JUnit, Mockito, Linux.

    \textbf{Lenoblles LOGKU (Russia, 2015--2016)}.
    Jan 2016 – \href{https://play.google.com/store/apps/details?id=com.altinntech.oopt_lo}{Lenoblles LOGKU}
    An Android application about the unique nature of the Leningrad region.
    Designed, developed, and deployed (on Google Play Market) a highly maintainable Android application
    (still on the market).
    Participated in testing and troubleshooting problems with the application.
    Java, Android, JUnit, Mockito, Gradle.

    \subsection*{Education}
    \begin{samepage}
        Master's Degree in Computer Science, 2013--2019\newline
        \href{https://etu.ru/en/university/}{Saint Petersburg Electrotechnical University "LETI"}, Russia\newline
        Faculty of Computer Technologies and Informatics, Department of Automation and Control Processes.
    \end{samepage}

    \subsection*{Hands-on Coding Expertise in Open Source Projects}
    \begin{itemize}
        \itemsep0em
        \item Contributor to \href{https://www.eolang.org}{EO} (a new programming language): Java, XML, XSLT, ANTLR4.
        \item Author of \href{https://github.com/volodya-lombrozo/jtcop}{jtcop} (a linter focused on test best practices): Java, Groovy, Maven.
        \item Author of \href{https://github.com/volodya-lombrozo/xnav}{xnav} (an XML navigation library): Java, XML, VTD-XML, ANTLR4, Saxon.
        \item Author of \href{https://github.com/volodya-lombrozo/jsmith}{jsmith} (a random generator for Java programs): Java, Maven, ANTLR4.
        \item Author of \href{https://github.com/volodya-lombrozo/jmh-benchmark-action}{jmh-benchmark-action} (a GitHub Action for JMH benchmarks): Groovy, GitHub API, JMH.
        \item Author of \href{https://github.com/volodya-lombrozo/hg-revision-plugin}{hg-revision-plugin} (a plugin for Mercurial): Java, Maven, Mercurial.
        \item Author of \href{https://github.com/volodya-lombrozo/conventional-commit-linter}{conventional-commit-linter} (a commit message linter): Java, Maven, Conventional Commits.
        \item Contributor to \href{https://www.rultor.com}{Rultor} (a DevOps assistant): Chatbot, Docker, GitHub API.
        \item Author of \href{https://github.com/volodya-lombrozo/newsman}{newsman} (a tool for tracking developer weekly activity): Ruby, GitHub API, OpenAI.
        \item Author of \href{https://github.com/volodya-lombrozo/aidy}{aidy} (an AI-assisted CLI for GitHub workflows): Go, GitHub API, OpenAI, DeepSeek.
        \item Author of \href{https://github.com/cqfn/refrax}{refrax} (a command-line tool for agentic refactoring of Java code): Go, A2A, OpenAI, DeepSeek.
    \end{itemize}

    \subsection*{Certifications}

    \begin{itemize}
        \itemsep0em
        \item \href{https://catalog-education.oracle.com/pls/certview/sharebadge?id=87F6A2FE819A5A5AF4120A05900AB28A461EE9A3EE9FBFA02721FADAEB3BCE19}{Oracle Certified Professional: Java SE 11 Developer, 2021}
        \item \href{https://www.credly.com/badges/e2d9ddda-20dc-433d-8ab7-18548fd0fd8f/public_url}{Oracle Certified Associate, Java SE 8 Programmer, 2020}
    \end{itemize}

    \subsection*{Public Not-for-profit Initiatives}
    \begin{itemize}
        \itemsep0em
        \item \href{https://www.kaicode.org}{KaiCode} open source festival: Organizer.
        KaiCode is an annual festival of open-source projects highlighting high-quality development.
    \end{itemize}

    \subsection*{Publications}
    \begin{itemize}
        \itemsep0em
        \item\emph{\href{https://arxiv.org/abs/2410.05631}{Embracing Objects Over Statics:
        An Analysis of Method Preferences
        in Open Source Java Frameworks}}, Oct 2024.
    \end{itemize}

    \subsection*{Patents}
    \begin{itemize}
        \itemsep0em
        \item\emph{\href{https://patentscope.wipo.int/search/en/detail.jsf?docId=WO2025084948&_cid=P20-ME9V9B-07633-1}{WO/2025/084948 Method of Obtaining Code Enclosing Entity and Related Device}}, filed Mar 2023, published Apr 2025.
    \end{itemize}

    \subsection*{Articles}
    \begin{itemize}
        \itemsep0em
        \item\emph{\href{https://dzone.com/articles/ai-do-you-trust-it}{AI: Do You Trust It?}}, Jul 24
        \item\emph{\href{https://dzone.com/articles/mastering-test-code-quality-assurance}{Mastering Test Code Quality
        Assurance}}, Feb 24
        \item\emph{\href{https://dzone.com/articles/thread-safety-pitfalls-in-xml-processing}{Thread-Safety Pitfalls in XML Processing}}, Feb 25
        \item\emph{\href{https://dev.to/volodya-lombrozo/from-vibe-coder-to-ai-assisted-architect-1nao}{From Vibe Coder to AI-Assisted Architect}}, May 7
        \item\emph{\href{https://dzone.com/articles/mock-the-file-system}{Mock the File System}}, Jun 25
    \end{itemize}

    \subsection*{Official Recommendations}
    \begin{itemize}
        \itemsep0em
        \item \href{https://github.com/volodya-lombrozo/volodya-lombrozo.github.io/blob/24eb2bbed8fac71c90ddef6ff4d93de8fb4f9f26/_cv/recommendations/letter_of_recommendation_ait.pdf}{LLC Alternative Innovative Technologies}.
    \end{itemize}

\end{document}
