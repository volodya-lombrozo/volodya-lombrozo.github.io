\documentclass{vl}

\begin{document}

    \vlPrintPhoto{}

    \section*{\Large \nospell{Vlad\'imir Zakh\'arov}}

    \href{mailto:volodya.lombrozo@gmail.com}{volodya.lombrozo@gmail.com}\\%
    \href{https://github.com/volodya-lombrozo}{GitHub} /
    \href{https://stackoverflow.com/users/10423604/volodya-lombrozo}{Stack Overflow} /
    \href{https://www.linkedin.com/in/vladimir-zakharov-lombrozo-b71744216/}{LinkedIn}

    \vspace*{12pt}

    \href{https://catalog-education.oracle.com/pls/certview/sharebadge?id=87F6A2FE819A5A5AF4120A05900AB28A461EE9A3EE9FBFA02721FADAEB3BCE19}{OCP}
    certified developer and architect with over 9 years of experience in engineering complex projects.
    Strong background in object-oriented programming, software testing, and code quality.
    Participated in many successful software projects, including for Huawei.
    \href{https://github.com/volodya-lombrozo?tab=repositories}{Creator} and
    \href{https://github.com/volodya-lombrozo}{active contributor} to several open-source projects,
    including\href{https://github.com/volodya-lombrozo/jtcop}{jtcop} a static linter focused on test best practices.
    One of the core developers of \textbf{\href{https://www.eolang.org}{EO}}, an object-oriented programming language.
    Regular contributor to the Stack Overflow community (profile 3K).
    Author of several patent applications and scientific publications.

    \subsection*{Recent Employments}

    \textbf{\href{https://www.huawei.com}{Huawei Technologies Co., Ltd.}} (Moscow, Russia), 10/2022--present.
    Researcher and Developer in the \textbf{Software Development Tools Cloud Technology Lab}.

    \begin{itemize}
        \item Core developer of \href{https://github.com/objectionary/eo}{EO}programming language;
        \item
        Creating \href{https://github.com/volodya-lombrozo/jtcop}{code quality}and developer assistance tools;
        \item Reviewing and presenting modern articles in the field of programming languages design and implementation;
        \item Providing technical seminars and workshops.
        \item Writing scientific articles and patents in the field of programming languages design and implementation.
    \end{itemize}

    \textbf{\href{https://www.huawei.com}{Huawei Technologies Co., Ltd.}} (Moscow, Russia), 12/2021--10/2022.
    Researcher and Developer in the \textbf{Database Intelligence and Optimization Technology Center}

    \begin{itemize}
        \item Developed an enterprise-wide profiling system for Huawei Cloud;
        \item Integrated profiling tools within large enterprise infrastructure;
        \item Researched modern profiling techniques and tools;
        \item Popularized code quality standards within the team.
    \end{itemize}

    \textbf{\href{https://altinntech.com/en/}{LLC Alternative Innovative Technologies}} (Saint Petersburg, Russia),
    07/2015--12/2021.
    Lead Software Developer.

    \begin{itemize}
        \item Created a fully-integrated, enterprise-wide complex information systems;
        \item Performed requirements analysis, design, implementation, testing, and maintenance of complex software
        systems;
        \item Configured monitoring and logging systems, and set up delivery systems (CI/CD);
        \item Popularized best practices in troubleshooting, monitoring, testing, and deploying Java applications
        within the company.
        \item Hands-on programming in Java;
        \item Led a team of several software developers.
    \end{itemize}

    \subsection*{Some Recent Projects}

    \subsubsection*{EO language (Russia, 2022--present)}
    A new object-oriented programming language
    that rethinks the very idea of objects and suggests a set of new OOP principles.
    Participated in the development of several core language components,
    including the compiler, runtime, and package manager.
    Improved code quality and testing coverage within the project.
    Fixed many performance issues and hard-to-detect concurrency bugs.
    Developed a translator and decompiler from bytecode to EO language.
    Wrote a patent application related to new methods of managing dependencies in programming languages.
    Wrote an article dedicated to the usage of OOP within Java projects.
    Participated in code reviews.
    Provided technical seminars and workshops within the team.
    Java, JS, EO, XSLT, XSD, XPath, ANTLR4, TeX, GitHub, and Open Source.

    \subsubsection*{Java Profiling System (Huawei, 2021--2022)}
    Enterprise-wide profiling system for Huawei Cloud.
    Developed and integrated profiling tools within large enterprise infrastructure.
    Researched modern profiling techniques and tools and integrated them into the system.
    Worked with the distributed international team of developers.
    Participated in code reviews.
    Java, Go, Mongodb, Docker, GitLab, Linux.

    \subsubsection*{Zillion Games (Russia, 2020)}
    Game platform for \href{http://zillion.games/}{Zillion Games}, that consists of various systems, including games,
    management and configuration systems, analytical systems, and payment processing systems.
    Developed most of the components and microservices.
    Troubleshot and fixed complex problems, including performance issues and hardly detectable concurrency bugs.
    Led a team of software developers.
    Participated in code reviews.
    Integrated code quality and testing practices in the team.
    Java, Microservices, Spring, Spring-AMQP, Hibernate, JDBC, Netty, RabbitMQ, JUnit, Mockito, Graylog, Zabbix, Docker,
    Bitbucket, Jenkins, SonarQube, Jira, Confluence.

    \subsubsection*{E.Clean car wash (Russia, Kazakhstan, 2019)}
    "E.Clean car wash" is an Android application
    that allows users to order a car wash based on their location.
    Designed and developed the backend part of the system.
    Participated in the development of the Android application.
    Participated in code reviews.
    Led a team of software developers.
    Built maintainable and efficient tests.
    Deployed the mobile app on the Play Store.
    Deployed and supported the rest of the system on dedicated Linux servers.
    Java, Jakarta EE, JSF, JPA, JAX-RS, Hibernate, MySQL, PrimeFaces, JUnit, Mockito, Android, Linux.

    \subsubsection*{Order Management System for "Staut" company (Russia, 2018)}
    The system manages the procurement of necessary components and equipment,
    allowing for managing lists of components and creating purchase orders in various formats.
    Designed and developed backend and frontend components of the system.
    Built a maintainable and efficient suite of tests.
    Participated in code reviews.
    Led a team of several software developers.
    Deployed and supported the system.
    Java, Java‐EE, JSF, JPA, Hibernate, MySQL, Apache-Poi, JUnit, Mockito, Jenkins, Linux.

    \subsubsection*{Analytical System for "Cytomed" Company (Russia, 2018)}
    Analytical System for \href{https://cytomed.ru/en/}{Cytomed}company that automates the process of tracking sales
    made by the company, it allows the creation of flexible analytical reports on the results of purchases and sales.
    Designed and developed the entire system.
    Built a maintainable and efficient layers of tests.
    Participated in code reviews.
    Lead a team of several software developers.
    Deployed and supported the system.
    Java, Java‐EE, Vaadin, JPA, Hibernate, MySQL, Jasper Reports, Apache-Poi, JUnit, Mockito, Jenkins, Linux.

    \subsubsection*{Terminal Network Management System (Russia, 2017)}
    Complex payment system that works with terminals that can accept payments, credit money, and debit funds,
    as well as issue cash banknotes from users' personal accounts.
    Designed and developed a web application for the management of remote terminal systems.
    Built a maintainable and efficient layer of tests that ensured stable and reliable operation of the system.
    Java, Java EE 7, JSF, PrimeFaces, JPA, Hibernate, MySQL, JUnit, Mockito, Linux.

    \subsubsection*{Lenoblles LOGKU (Russia, 2015--2016)}
    Jan 2016 – \href{https://play.google.com/store/apps/details?id=com.altinntech.oopt_lo}{Lenoblles LOGKU}
    An Android application about the unique nature of the Leningrad region.
    Designed and developed a highly maintainable Android application (still on the market).
    Participated in testing and troubleshooting problems with the application.
    Deployed the application on Google Play Market.
    Java, Android, JUnit, Mockito, Gradle.

    \subsection*{Hands-on Coding Expertise (in the open-source)}

    Contributor to \href{https://www.eolang.org}{EO} (new programming language):
    Java, XML, XSLT, ANTLR4.

    Author of  \href{https://github.com/volodya-lombrozo/jtcop}{jtcop} (focused on test best practices):
    Java, Groovy, Maven.

    Author of \href{https://github.com/volodya-lombrozo/hg-revision-plugin}{hg-revision-plugin} (mercurial plugin):
    Java, Maven, Mercurial.

    Author of \href{https://github.com/volodya-lombrozo/conventional-commit-linter}{conventional-commit-linter} (commit linter):
    Java, Maven, Conventional Commits.

    Contributor to \href{https://www.rultor.com}{Rultor} (DevOps assistant):
    Chatbot, Docker, GitHub API.

    \subsection*{Certifications}

    \begin{itemize}
        \item \href{https://catalog-education.oracle.com/pls/certview/sharebadge?id=87F6A2FE819A5A5AF4120A05900AB28A461EE9A3EE9FBFA02721FADAEB3BCE19}{Oracle Certified Professional: Java SE 11 Developer, 2021}
        \item \href{https://www.credly.com/badges/e2d9ddda-20dc-433d-8ab7-18548fd0fd8f/public_url}{Oracle Certified Associate, Java SE 8 Programmer, 2020}
    \end{itemize}

    \subsection*{Education}

    \begin{samepage}
        Master's Degree in Computer Science, 2013--2019\newline
        \href{https://etu.ru/en/university/}{Saint Petersburg Electrotechnical University "LETI"}, Russia\newline
        Faculty of Computer Technologies and Informatics, Department of Automation and Control Processes.
    \end{samepage}

    \subsection*{Public Not-for-profit Initiatives}
    \begin{itemize}
        \item \href{https://www.kaicode.org}{KaiCode} open source festival: Organizer.
        KaiCode is an annual festival of open-source projects highlighting high-quality development.
    \end{itemize}
    \subsection*{Official Recommendations}
    \begin{itemize}
        \item \href{https://github.com/volodya-lombrozo/volodya-lombrozo.github.io/blob/24eb2bbed8fac71c90ddef6ff4d93de8fb4f9f26/_cv/recommendations/letter_of_recommendation_ait.pdf}{LLC Alternative Innovative Technologies}.
    \end{itemize}

\end{document}